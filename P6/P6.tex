% Preamble
\documentclass[12pt, letterpaper]{article}

% Packages
\usepackage[utf8]{inputenc}
\usepackage[letterpaper, portrait, margin=1in]{geometry}
\usepackage{listings}
\usepackage{hyperref}
% End Packages

\title{CS 373S P6: Create and Undo Visitors Using R3}
\author{
	Jeff Bell - jhb2596 \\ \underline{\href{mailto:jhbell@utexas.edu}{jhbell@utexas.edu}}
	\and
	John Filippone - jpf735  \\ \underline{\href{mailto:johnpfilippone@yahoo.com}{johnpfilippone@yahoo.com}}
}
\date{\today}

% Document
\begin{document}
\maketitle

% About
\section*{Locating the Files}
The two zip files (\texttt{MakeVisitorUsingR3.zip} and \texttt{UndoVisitorUsingR3.zip})
containing the edited eclipse projects can be found in the \texttt{filippone-bell} directory.

Each project zip has a PDF file containing an explanation of the R3 script contained in 
the project. These PDFs are called \texttt{MakeExplanation.pdf} and \texttt{UndoExplanation.pdf}
respectively.

\section*{Issues With R3}
On MacOS, I had to run the R3 version of Eclipse from the command line. This involved navigating
into the \texttt{Eclipse.app} application and running the executable binary with \texttt{./eclipse}.

The given version of R3, although limited, was clearly explained in documentation. The workflow took
some getting used to and is currently not easy to use. Better integration with eclipse or complete separation from
eclipse would be helpful in improving the workflow.

\end{document}